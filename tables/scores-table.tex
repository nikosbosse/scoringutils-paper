% !TeX root = ../article.Rnw
% Please add the following required packages to your document preamble:
% \usepackage{booktabs}
\begin{sidewaystable}[]

% \begin{adjustbox}{angle=90}

\centering
\begin{tabularx}{\linewidth}{|X|X|X|X|}



\hline
Metric                                                                  & Formula                                                                                                                                                                                                                                                                                                                                                                                                                                                                                                                                                                                                                                                                                                                                                                                                                                                                                                         & Application and Explanation                                                                                                                                                                                                                                                                                                                                                                                                                    & Caveats                                                                                                                                                        \\ \hline
WIS (Weighted) interval score                                           & For a single interval, the score is computed as $IS_\alpha(F,y) = (u-l) + \frac{2}{\alpha} \cdot (l-y) \cdot 1(y \leq l) + \frac{2}{\alpha} \cdot (y-u) \cdot 1(y \geq u)$, where 1() is the indicator function, y is the true value, and $l$ and $u$ are the $\frac{\alpha}{2}$ and $1 - \frac{\alpha}{2}$ quantile of $F$, i.e. the lower and upper bound of a single prediction interval. For a set of $K$ prediction intervals and the median $m$, the score is computed as a weighted sum, $WIS = \frac{1}{K + 0.5} \cdot (w_0 \cdot |y - m| + \sum_{k = 1}^{K} w_k \cdot IS_{\alpha}(F, y))$. $w_k$ is a weight for every interval. Usually, $w_k = \frac{\alpha_k}{2}$ and $w_0 = 0.5$                                                                                                                                                                                                                   & proper scoring rule for quantile forecasts. converges to crps for increasing number of interval. The score can be decomposed into a sharpness contribution and penalties for over- and underprediction.                                                                                                                                                                                                                                        & The wis is based on measures of absolute error. When averaging across multiple targets, it will therefore be dominated by targets with higher absolute values. \\
CRPS (Continuous) ranked probability score                              & $\text{CRPS}(F, y) = \int_{-\infty}^\infty \left( F(x) - \mathbb{1}(x \geq y) \right)^2 dx$, where y is the true observed value an F the predictive distribution. Another represenation allows for easier computation using samples. Take $ \text{CRPS}(F, y) = \frac{1}{2} \mathbb{E}_{F} |X - X'| - \mathbb{E}_P |X - y|$, where $X$ and $X'$ are independent realisations from the predictive distributions $F$ with finite first moment and $y$ is the true value. We can simply replace $X$ and $X'$ by samples sum over all possible combinations to obtain the CRPS.  For integer-valued forecasts, the RPS is given as $ \text{RPS}(F, y) = \sum_{x = 0}^\infty (F(x) - \mathbb{1} (x \geq y))^2. $                                                                                                                                                                                                     & proper scoring rule for continuous and integer forecasts. Implemented for sample-based forecasts.                                                                                                                                                                                                                                                                                                                                              & Also sensitive to the absolute value of the predicted quantity                                                                                                 \\
\begin{tabular}[c]{@{}l@{}}DSS\\ \\ Dawid-Sebastiani score\end{tabular} & 2.257546e+02                                                                                                                                                                                                                                                                                                                                                                                                                                                                                                                                                                                                                                                                                                                                                                                                                                                                                                    & proper scoring rule for continuous and integer forecasts. The dss has a slightly simpler formula that only relies on the first moments of the predictive distribution. If in doubt, we would recommend the crps, but the difference should not be large in practice.                                                                                                                                                                           &                                                                                                                                                                \\
log score                                                               & The Log score is given by $ \text{log score} = \log f(y)$, where $f$ is the predictive density function and y is the true value.                                                                                                                                                                                                                                                                                                                                                                                                                                                                                                                                                                                                                                                                                                                                                                                & \begin{tabular}[c]{@{}l@{}}The log score is a proper scoring rule best suited for continuous forecasts. While in principle applicable to integer valued forecasts, its computation requires a density function which is in practice difficult for integer valued forecasts. \\ One concern with the log score is that it can assign extreme scores if the predictive distribution gives little probability to the observed event.\end{tabular} & Sensitive to outliers. Individual log score contributions can go to infinity                                                                                   \\
Brier score                                                             & $\text{Brier Score} = \frac{1}{N} \sum_{n = 1}^{N} (\text{prediction}_n - \text{outcome}_n)^2$, where prediction                                                                                                                                                                                                                                                                                                                                                                                                                                                                                                                                                                                                                                                                                                                                                                                                & Proper scoring rule for binary forecasts                                                                                                                                                                                                                                                                                                                                                                                                       &                                                                                                                                                                \\
interval coverage                                                       & (Interval) coverage for a single prediction interval can be calculated as $IC_\alpha = \text{nominal coverage} - \text{actual empirical coverage}$, where nominal coverage is $1 - \alpha$ and empirical coverage is the percentage of true values actually covered by the $1 - \alpha$ prediction intervals. Interval coverage can then be aggregated over all interval levels: $\text{Coverage deviation} = \frac{1}{K} \sum_{k = 1}^{K} IC_{\alpha_k}$                                                                                                                                                                                                                                                                                                                                                                                                                                                         &                                                                                                                                                                                                                                                                                                                                                                                                                                                &                                                                                                                                                                \\




\end{tabularx}

% \end{adjustbox}

\end{sidewaystable}